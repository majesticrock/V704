\section{Diskussion}
\label{sec:Diskussion}

\subsection{Absorption von Gamma-Strahlung}

Die rechnerisch bestimmten Werte für $N_0$ sind $N_{0,\text{Eisen}} = (151 \pm 2)$ $\frac{1}{\symup{s}}$ und $N_{0,\text{Blei}} = (156 \pm 10)$ $\frac{1}{\symup{s}}$. 
Beide liegen sehr nah an dem experimentell bestimmten Wert von $N_{0,\text{Theorie}} (156 \pm 12)$ $\frac{1}{\symup{s}}$ und auch innerhalb dessen Standardabweichung.

Die Absorptionskoeffizienten von Blei und Eisen sind

\begin{align*}
    \mu_\text{Blei} &= (1,03 \pm 0,02) \, \frac{1}{\symup{cm}}\\
    \mu_\text{Eisen} &= (0,431 \pm 0,005) \, \frac{1}{\symup{cm}}.
\end{align*}

Diese weichen teils stark von den Theoriewerten

\begin{align*}
    \mu_\text{Blei, Theorie} &= 0,69 \, \frac{1}{\symup{cm}}\\
    \mu_\text{Eisen, Theorie} &= 0,241 \, \frac{1}{\symup{cm}}
\end{align*}

ab. Für Blei ist diese $\sim \! 50\%$ und für Eisen $\sim \! 24\%$.
Die Messwerte liegen jedoch alle sehr gut auf der Ausgleichsgeraden und die hohe Aufnahmedauer lässt statistische Fehler gering werden.
Daher werden die Messwerte als gut angenommen.

Jedoch sind die Theoriewerte selbst nur eine Näherung und sie betrachten lediglich die Abschirmung durch den Comptoneffekt.
Allerdings wirken auch andere Abschirmungen und die Näherung ist nicht exakt, was die große Abweichung erklärt.



\subsection{Absorption von Beta-Strahlung}

Die hohe Aufnahmedauer vermindert die statistischen Fehler stark. Dennoch sind durch die geringe Anzahl von gemessenen Teilchen pro Zeiteinheit
verglichen mit der Gamma-Strahlung hohe prozentuale Abweichungen zu erwarten.
Für $R_\text{max} = (0,7 \pm 0,1)$ $\frac{\symup{g}}{\symup{cm}^2}$ beträgt diese $\sim \! 18 \%$ und für $E_\text{max} = (1,6 \pm 0,3)$ MeV $\sim \! 16,3 \%$.

Die linearen Fits nähern die Werte innerhalb ihrer eigenen Standardabweichung gut an. 
Es fällt jedoch auf, dass die Gerade bei der Gamma-Strahlung die Messwerte sehr viel besser annähert,
was wie oben bereits erwähnt, an der großen Diskrepanz zwischen den gemessenen Teilchen pro Zeiteinheit liegt.