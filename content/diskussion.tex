\section{Diskussion}
\label{sec:Diskussion}

\subsection{Absorption von Gamma-Strahlung}

Die rechnerisch bestimmten Werte für $N_0$ sind $(151 \pm 2)$ $\frac{1}{\symup{s}}$ und $N_0 = (156 \pm 10)$ $\frac{1}{\symup{s}}$. 
Beide liegen sehr nah an dem experimentell bestimmten Wert von $(156 \pm 12)$ $\frac{1}{\symup{s}}$ und auch innerhalb dessen Standardabweichung.

Die Absorptionskoeffizienten von Blei und Eisen sind

\begin{align*}
    \mu_\text{Blei} &= (1,03 \pm 0,02) \, \frac{1}{\symup{cm}}\\
    \mu_\text{Eisen} &= (0,431 \pm 0,005) \, \frac{1}{\symup{cm}}.
\end{align*}

Diese weichen teils stark von den Theoriewerten

\begin{align*}
    \mu_\text{Blei, Theorie} &= 0,69 \, \frac{1}{\symup{cm}}\\
    \mu_\text{Eisen, Theorie} &= 0,241 \, \frac{1}{\symup{cm}}
\end{align*}

ab. Für Blei ist diese $\sim \! 50\%$ und für Eisen $\sim \! 24\%$.
Die Messwerte liegen jedoch alle sehr gut auf der Ausgleichsgeraden und die hohe Aufnahmedauer lässt statistische Fehler gering werden.
Daher werden die Messwerte als gut angenommen.

Jedoch sind die Theoriewerte selbst nur eine Näherung und sie betrachten lediglich die Abschirmung durch den Comptoneffekt.
Allerdings wirken auch eine Abschirmungen und es die Näherung ist nicht exakt, was die große Abweichung erklärt.