\section{Auswertung}
\label{sec:Auswertung}

\subsection{Absoprtion von Gamma-Strahlung}

Aus GLEICHUNG erfolgt der Zusammenhang

\begin{equation}
    \label{eqn:plot}
    f(D) := \ln \bigg (\frac{ N \cdot 1 \symup{s}}{t} - \Delta N \cdot 1 \symup{s} \bigg) = - \mu \cdot D - \ln (N_0 \cdot 1 \symup{s}),
\end{equation}

wobei

\begin{center}
    $\Delta N = \frac{829}{900 \symup{s}}$
\end{center}

die Zählrate pro Sekunde ist, die gemessen wird, wenn weder Absorber noch Quelle in der Messvorrichtung sind.

Zur Bestimmung des Absorptionskoeffizienten von Blei werden nun die Daten aus \autoref{tab:blei} betrachet.

\begin{table}[!htp]
\centering
\caption{Gesamtzählraten mit Blei als Abschirmmaterial bei verschiedenen Dicken.}
\label{tab:blei}
\begin{tabular}{S[table-format=1.1] S[table-format=3.0] S[table-format=5.0] @{${}\pm{}$} S[table-format=3.0]}
\toprule
{$D$ / cm} & {$t$ / s} & \multicolumn{2}{c}{Zählrate $N$} \\
\midrule
5.0 & 600 &   985 &  31 \\
4.5 & 540 &  1495 &  38 \\
4.0 & 480 &  1850 &  43 \\
3.5 & 420 &  2212 &  47 \\
3.0 & 360 &  2764 &  52 \\
2.5 & 300 &  3936 &  62 \\
2.0 & 240 &  4877 &  69 \\
1.5 & 180 &  6696 &  81 \\
1.0 & 120 &  6513 &  80 \\
0.5 &  60 &  5531 &  74 \\
0.0 & 100 & 15664 & 125 \\
\bottomrule
\end{tabular}
\end{table}

Diese werden nach der in Gleichung \eqref{eqn:plot} in einem Plot aufgetragen.
Mittels Python 3.7.0 wird eine lineare Ausgleichsrechnung durchgeführt, die der obigen Formel genügt.
Die entstehende Gerade wird ebenfalls im Plot eingefügt.
Dieser ist in \autoref{fig:plot_blei} zu sehen.

\begin{figure}
    \centering
    \includegraphics[width=0.95\textwidth]{build/plot_blei.pdf}
    \caption{Plot und Fit der Messwerte mit Blei als Abschirmmaterial zur Bestimmung des Absorptionskoeffizienten.}
    \label{fig:plot_blei}
\end{figure}

Die Koeffizienten ergeben sich zu 

\begin{center}
    $\mu_\text{Blei} = (1,03 \pm 0,02)$ $\frac{1}{\symup{cm}}$

    $\ln ( N_ 0 \cdot 1 \symup{s}) = (5,05 \pm 0,07).$
\end{center}

Damit ist $N_0 = (156 \pm 10)$ $\frac{1}{\symup{s}}$.
Über GLEICHUNG kann ein Theoriewert für den Absorptionskoeffizienten als

\begin{center}
    $\mu_\text{Blei, Theorie} = 0,69$ $\frac{1}{\symup{cm}}$
\end{center}

bestimmt werden.
Dabei ist mittels GLEICHUNG

\begin{center}
    $\sigma (\epsilon ) = 2,5655 \cdot 10^{-29}$ $\frac{1}{\symup{m}^2}$,
\end{center}

wobei $\epsilon = 1,295$ beträgt \cite{V704}.

Zur Berechnung des Absorptionskoeffizienten von Eisen wird analog vorgegangen.

\begin{table}[!htp]
\centering
\caption{Gesamtzählraten mit Eisen als Abschirmmaterial bei verschiedenen Dicken.}
\label{tab:eisen}
\begin{tabular}{S[table-format=1.1] S[table-format=3.0] S[table-format=5.0] @{${}\pm{}$} S[table-format=3.0]}
\toprule
{$D$ / cm} & {$t$ / s} & \multicolumn{2}{c}{Zählrate $N$} \\
\midrule
5.0 & 300 &  5695 &  75 \\
4.5 & 270 &  6010 &  77 \\
4.0 & 240 &  6640 &  81 \\
3.5 & 210 &  7426 &  86 \\
3.0 & 180 &  7937 &  89 \\
2.5 & 150 &  7795 &  88 \\
2.0 & 120 &  7695 &  87 \\
1.5 &  90 &  6977 &  83 \\
1.0 &  60 &  5885 &  76 \\
0.5 &  50 &  6297 &  79 \\
0.0 & 100 & 15664 & 125 \\
\bottomrule
\end{tabular}
\end{table}

Die Werte in \autoref{tab:eisen} werden in einem Plot gemäß Gleichung \eqref{eqn:plot} aufgetragen und durch diese eine Ausgleichsgerade gelegt.
Der entstehende Plot ist in \autoref{fig:plot_eisen} zu sehen.

\begin{figure}
    \centering
    \includegraphics[width=0.95\textwidth]{build/plot_eisen.pdf}
    \caption{Plot und Fit der Messwerte mit Eisen als Abschirmmaterial zur Bestimmung des Absorptionskoeffizienten.}
    \label{fig:plot_eisen}
\end{figure}

Die Koeffizienten berechnen sich mittels Python 3.7.0 als

\begin{center}
    $\mu_\text{Eisen} = (0,431 \pm 0,005)$ $\frac{1}{\symup{cm}}$

    $\ln ( N_ 0 \cdot 1 \symup{s}) = (5,02  \pm 0,01).$
\end{center}

Womit hier $N_0 = (151 \pm 2)$ $\frac{1}{\symup{s}}$ ist.

Der Theoriewert wird mit demselben $\sigma$ wie schon bei Blei und ebenfalls mittels GLEICHUNG als

\begin{center}
    $\mu_\text{Eisen, Theorie} = 0,568$ $\frac{1}{\symup{cm}}$
\end{center}

bestimmt.



\subsection{Absoprtion von Beta-Strahlung}

Zur Bestimmung der Maximalenergie werden zunächst die Werte aus \autoref{tab:beta} in einem Plot dargestellt.
Dabei wird $\ln(A \cdot 1 \symup{s})$ gegen die Massenbelegung $R$ aufgetragen, welche sich aus GLEICHUNG berechnet.
Anschließend wird durch die beiden linearen Teile,
also zwischen den $R$-Werten von $2,70$ $\frac{\symup{g}}{\symup{cm}^2}$ und $0,683$ $\frac{\symup{g}}{\symup{cm}^2}$, 
sowie $0,683$ $\frac{\symup{g}}{\symup{cm}^2}$ und $1,301$ $\frac{\symup{g}}{\symup{cm}^2}$ je eine Ausgleichgsgerade der Form

\begin{equation}
    \ln(A \cdot 1 \symup{s}) = a_i x + b_i
\end{equation}

gelegt, wobei $a_i$ und $b_i$ die Parameter zu den Geraden 1 und 2 sind.

\begin{table}[!htp]
  \centering
  \caption{Aufgenommene Zählraten zur Abschirmung von Betastrahlung durch Aluminiumplatten für verschiedene Zeiten.}
  \label{tab:beta}
  \begin{tabular}{ccccc}
    \toprule
    $d$ / µm & $R$ / $\frac{\symup{g}}{\symup{cm}^2}$ &  $\Delta t$ / s & $N$ & $A$ / $\frac{1}{\symup{s}}$\\
    \midrule
    482 \pm 1,0 & 1,301 \pm 0,003 & 1100 &  755 \pm 27,48 &  0,69 \pm 0,03 \\
    444 \pm 2,0 & 1,200 \pm 0,005 & 1000 &  660 \pm 25,69 &  0,66 \pm 0,03 \\
    400 \pm 1,0 & 1,080 \pm 0,003 &  900 &  580 \pm 24,08 &  0,64 \pm 0,03 \\
    338 \pm 5,0 & 0,913 \pm 0,014 &  800 &  525 \pm 22,91 &  0,66 \pm 0,03 \\
    302 \pm 1,0 & 0,815 \pm 0,003 &  700 &  524 \pm 22,89 &  0,75 \pm 0,03 \\
    253 \pm 1,0 & 0,683 \pm 0,003 &  600 &  477 \pm 21,84 &  0,78 \pm 0,04 \\
    200 \pm 1,0 & 0,540 \pm 0,003 &  500 &  881 \pm 29,68 &  1,76 \pm 0,06 \\
    160 \pm 1,0 & 0,432 \pm 0,003 &  400 & 1795 \pm 42,37 &  4,49 \pm 0,11 \\
    153 \pm 0,5 & 0,413 \pm 0,001 &  300 & 2083 \pm 45,64 &  6,94 \pm 0,15 \\
    125 \pm 0,0 & 0,338 \pm 0,000 &  200 & 1273 \pm 35,68 &  6,37 \pm 0,18 \\
    100 \pm 0,0 & 0,270 \pm 0,000 &  100 & 2792 \pm 52,84 & 27,92 \pm 0,53 \\
    \bottomrule
  \end{tabular}
\end{table}


\begin{figure}
    \centering
    \includegraphics[width=0.95\textwidth]{build/plot_beta.pdf}
    \caption{Plot und Fit der Aktivität gegen die Massenbelegung bei Beta-Strahlung.}
    \label{fig:plot_beta}
\end{figure}

Mittels Python 3.7.0 lassen sich diese Koeffizienten als

\begin{align*}
    a_1 &= (-9 \pm 1) \, \frac{\symup{cm}^2}{\symup{g}}\\
    b_1 &= 5,5 \pm 0,8
\end{align*}

und

\begin{align*}
    a_2 &= (0,1 \pm 0,1) \, \frac{\symup{cm}^2}{\symup{g}}\\
    b_2 &= -0,5 \pm 0,1
\end{align*}

bestimmen. Der entsprechende Plot ist in \autoref{fig:plot_beta} zu sehen.

Die maximale Reichweite $R_\text{max}$ ist der Schnittpunkt der beiden Gerade, welcher mittels

\begin{center}
    $R_\text{max} = \frac{b_2 - b_1}{a_1 - a_2} = (0,7 \pm 0,2)$ $\frac{\symup{g}}{\symup{cm}^2}$
\end{center}

bestimmt werden kann. Mittels GLEICHUNG kann daraus die Maximalenergie als

\begin{center}
    $E_\text{max} = (1,4 \pm 0,3)$ MeV
\end{center}

errechnet werden.