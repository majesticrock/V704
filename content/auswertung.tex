\section{Auswertung}
\label{sec:Auswertung}

\subsection{Absoprtion von Gamma-Strahlung}

Aus GLEICHUNG erfolgt der Zusammenhang

\begin{equation}
    \label{eqn:plot}
    f(D) := \ln \bigg (\frac{ N \cdot 1 \symup{s}}{t} - \Delta N \cdot 1 \symup{s} \bigg) = - \mu \cdot D - \ln (N_0 \cdot 1 \symup{s}),
\end{equation}

wobei

\begin{center}
    $\Delta N = \frac{829}{900 \symup{s}}$
\end{center}

die Zählrate pro Sekunde ist, die gemessen wird, wenn weder Absorper noch Quelle in der Messvorrichtung sind.

Zur Bestimmung des Absorptionskoeffizienten von Blei werden nun die Daten aus \autoref{tab:blei} betrachet.

\begin{table}[!htp]
\centering
\caption{Gesamtzählraten mit Blei als Abschirmmaterial bei verschiedenen Dicken.}
\label{tab:blei}
\begin{tabular}{S[table-format=1.1] S[table-format=3.0] S[table-format=5.0] @{${}\pm{}$} S[table-format=3.0]}
\toprule
{$D$ / cm} & {$t$ / s} & \multicolumn{2}{c}{Zählrate $N$} \\
\midrule
5.0 & 600 &   985 &  31 \\
4.5 & 540 &  1495 &  38 \\
4.0 & 480 &  1850 &  43 \\
3.5 & 420 &  2212 &  47 \\
3.0 & 360 &  2764 &  52 \\
2.5 & 300 &  3936 &  62 \\
2.0 & 240 &  4877 &  69 \\
1.5 & 180 &  6696 &  81 \\
1.0 & 120 &  6513 &  80 \\
0.5 &  60 &  5531 &  74 \\
0.0 & 100 & 15664 & 125 \\
\bottomrule
\end{tabular}
\end{table}

Diese werden nach der in Gleichung \eqref{eqn:plot} in einem Plot aufgetragen.
Mittels Python 3.7.0 wird eine lineare Ausgleichsrechnung durchgeführt, die der obigen Formel genügt.
Die entstehende Gerade wird ebenfalls im Plot eingefügt.
Dieser ist in \autoref{fig:plot_blei} zu sehen.

\begin{figure}
    \centering
    \includegraphics[width=0.95\textwidth]{build/plot_blei.pdf}
    \caption{Plot und Fit der Messwerte mit Blei als Abschirmmaterial zur Bestimmung des Absorptionskoeffizienten.}
    \label{fig:plot_blei}
\end{figure}

Die Koeffizienten ergeben sich zu 

\begin{center}
    $\mu_\text{Blei} = (1.03 \pm 0.02)$ $\frac{1}{\symup{cm}}$

    $\ln ( N_ 0 \cdot 1 \symup{s}) = (5.05 \pm 0.07).$
\end{center}

Damit ist $N_0 = (156 \pm 10)$ $\frac{1}{\symup{s}}$.
Über GLEICHUNG kann ein Theoriewert für den Absorptionskoeffizienten als

\begin{center}
    $\mu_\text{Blei, Theorie} = 0.69$ $\frac{1}{\symup{cm}}$
\end{center}

bestimmt werden.
Dabei ist mittels GLEICHUNG

\begin{center}
    $\sigma (\epsilon ) = 2.5655 \cdot 10^{-29}$ $\frac{1}{\symup{m}^2}$,
\end{center}

wobei $\epsilon = 1.295$ beträgt \cite{V704}.

Zur Berechnung des Absorptionskoeffizienten von Eisen wird analog vorgegangen.

\begin{table}[!htp]
\centering
\caption{Gesamtzählraten mit Eisen als Abschirmmaterial bei verschiedenen Dicken.}
\label{tab:eisen}
\begin{tabular}{S[table-format=1.1] S[table-format=3.0] S[table-format=5.0] @{${}\pm{}$} S[table-format=3.0]}
\toprule
{$D$ / cm} & {$t$ / s} & \multicolumn{2}{c}{Zählrate $N$} \\
\midrule
5.0 & 300 &  5695 &  75 \\
4.5 & 270 &  6010 &  77 \\
4.0 & 240 &  6640 &  81 \\
3.5 & 210 &  7426 &  86 \\
3.0 & 180 &  7937 &  89 \\
2.5 & 150 &  7795 &  88 \\
2.0 & 120 &  7695 &  87 \\
1.5 &  90 &  6977 &  83 \\
1.0 &  60 &  5885 &  76 \\
0.5 &  50 &  6297 &  79 \\
0.0 & 100 & 15664 & 125 \\
\bottomrule
\end{tabular}
\end{table}

Die Werte in \autoref{tab:eisen} werden in einem Plot gemäß Gleichung \eqref{eqn:plot} aufgetragen und durch diese eine Ausgleichsgerade gelegt.
Der entstehende Plot ist in \autoref{fig:plot_eisen} zu sehen.

\begin{figure}
    \centering
    \includegraphics[width=0.95\textwidth]{build/plot_eisen.pdf}
    \caption{Plot und Fit der Messwerte mit Eisen als Abschirmmaterial zur Bestimmung des Absorptionskoeffizienten.}
    \label{fig:plot_eisen}
\end{figure}

Die Koeffizienten berechnen sich mittels Python 3.7.0 als

\begin{center}
    $\mu_\text{Eisen} = (0.431 \pm 0.005)$ $\frac{1}{\symup{cm}}$

    $\ln ( N_ 0 \cdot 1 \symup{s}) = (5.02  \pm 0.01).$
\end{center}

Womit hier $N_0 = (151 \pm 2)$ $\frac{1}{\symup{s}}$ ist.

Der Theoriewert wird mit demselben $\sigma$ wie schon bei Blei und ebenfalls mittels GLEICHUNG als

\begin{center}
    $\mu_\text{Eisen, Theorie} = 0.568$ $\frac{1}{\symup{cm}}$
\end{center}

bestimmt.